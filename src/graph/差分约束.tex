\textbf{差分约束系统} 是一种特殊的 $n$ 元一次不等式组 。

差分约束系统中的每个约束条件 $x_i-x_j\le c_k$ 都可以变形成 $x_i\le x_j+c_k$ 与 $x_j\ge x_i-c_k$ ,这与单源最短路中的三角形不等式非常相似。因此,我们可以把每个变量 $x_i$ 看做图中的一个结点,对于每个约束条件连边。

需要注意的是,有些题目看能会对解的上、下界进行约束,因此我们需要对这些条件处理(这里只考虑对于这 $n$ 个元素 \textbf{只约束了上界} 或 \textbf{只约束了下界} ):

\begin{itemize}
    \item 只约束下界:有 $0$ 号点向每一个点连一条长为 $Lim_i$ 的边,表示第 $i$ 号元素的 \textbf{下界} 为 $Lim_i$ ,如图所示建边:
    \begin{center}
        \begin{tabular}{|c|c|c|}
            \hline
            题意 & 转化 & 连边 \\
            \hline
            $x_a-x_b\ge c$ & $x_a\ge x_b+c$ & \texttt{add(b,a,c)} \\
            $x_a-x_b\le c$ & $x_b\ge x_a-c$ & \texttt{add(a,b,-c)} \\
            $x_a=x_b$ & $x_a\ge x_b$ , $x_b\le x_a$ & \texttt{add(a,b,0),add(b,a,0)} \\
            \hline
        \end{tabular}
    \end{center}
    之后对整张图跑 \textbf{最长路} 。
    \item 只约束上界:有 $0$ 号点向每一个点连一条长为 $Lim_i$ 的边,表示第 $i$ 号元素的 \textbf{上界} 为 $Lim_i$ ,如图所示建边:
    \begin{center}
        \begin{tabular}{|c|c|c|}
            \hline
            题意 & 转化 & 连边 \\
            \hline
            $x_a-x_b\ge c$ & $x_b\le x_a-c$ & \texttt{add(a,b,-c)} \\
            $x_a-x_b\le c$ & $x_a\le x_b+c$ & \texttt{add(b,a,c)} \\
            $x_a=x_b$ & $x_a\le x_b$ , $x_b\le x_a$ & \texttt{add(a,b,0),add(b,a,0)} \\
            \hline
        \end{tabular}
    \end{center}
    之后对整张图跑 \textbf{最短路} 。
\end{itemize}

设 $dist[0]=0$ ,若存在负环 $/$ 正环,则不等式无解,否则 $x_i=dist[i]$ 是该差分约束系统的一组解 。

最坏情况下(存在负环 $/$ 正环)复杂度为 $O(nm)$ 。

\textbf{注意:整个图不一定是联通的!} 