\documentclass[10pt, a4paper, oneside]{ctexart}
\usepackage{minted, multicol, geometry, graphicx, fancyvrb, amssymb}
\usepackage{relsize, setspace, enumitem, float, hyperref, amsmath}
\usepackage[table,xcdraw]{xcolor} % 表格背景颜色
\usepackage{comment} 

\geometry{a4paper, scale = 0.85}

\setenumerate[1]{itemsep=0pt,partopsep=0pt,parsep=\parskip,topsep=0pt}
\setitemize[1]{itemsep=0pt,partopsep=0pt,parsep=\parskip,topsep=0pt}
\setdescription{itemsep=0pt,pa
rtopsep=0pt,parsep=\parskip,topsep=0pt}
\setlength{\parindent}{0pt}
\setlength{\columnsep}{18pt}

\setminted[cpp]{
	style=xcode,
	mathescape,
	linenos,
	autogobble,
	baselinestretch=1,
	tabsize=3,
	fontsize=\scriptsize,
	%bgcolor=Gray,
	frame=single,
	framesep=1mm,
	framerule=0.3pt,
	numbersep=1mm,
	breaklines=true,
	breaksymbolsepleft=2pt,
	%breaksymbolleft=\raisebox{0.8ex}{ \small\reflectbox{\carriagereturn}}, %not moe!
	%breaksymbolright=\small\carriagereturn,
	breakbytoken=false,
	% showtabs=true,
	% tab={\relscale{0.6} $\big\vert \ \ \ $ \relscale{1}},
}

\setminted[bash]{
    style=xcode,
	mathescape,
	linenos,
	autogobble,
	baselinestretch=1,
	tabsize=3,
	fontsize=\scriptsize,
	frame=single,
	framesep=1mm,
	framerule=0.3pt,
	numbersep=1mm,
	breaklines=true,
	breaksymbolsepleft=2pt,
	breakbytoken=false,
}
\setminted[python3]{
    style=xcode,
	mathescape,
	linenos,
	autogobble,
	baselinestretch=1,
	tabsize=3,
	fontsize=\scriptsize,
	frame=single,
	framesep=1mm,
	framerule=0.3pt,
	numbersep=1mm,
	breaklines=true,
	breaksymbolsepleft=2pt,
	breakbytoken=false,
}

\title{ACM 模板}
\author{黄佳瑞,钱智煊,林乐逍}
\date{\today}

\begin{document}
    \scriptsize
    \maketitle
    \newpage
    
    \begin{multicols}{2}
        \tableofcontents
        \newpage

        \section{做题指导}
        \subsection{上机前你应该注意什么}
        \input{src/guide/before-code.tex}
        \subsection{机上你应该注意什么}
        \input{src/guide/on-code.tex}
        \subsection{交题前你应该注意什么}
        \input{src/guide/after-code.tex}
        \subsection{如果你的代码挂了}
        \input{src/guide/debug.tex}
        \subsection{另外一些注意事项}
        \input{src/guide/else.tex}
        \subsection{阴间错误集锦}
        \begin{itemize}
    \item 多测不清空。
    \item 该开 \verb|long long| 的地方不开 \verb|long long| 。一般情况下建议直接 \verb|#define int long long| 。
    \item 注意变量名打错。例如 u 打成 v 或 a 打成 t 。建议读代码的时候专门检查此类错误。
    \item 树剖,倍增LCA之类的 \verb|while| 打成 \verb|if|。
\end{itemize}
        \newpage

        \section{图论}
        \subsection{Tarjan 边双、点双(圆方树)}
        \input{src/graph/tarjan.tex}
        \subsection{O(1) LCA}
        \input{src/graph/o1lca.tex}

        % 在线 O(1) LCA ?

        \subsection{某些路径问题单log做法}
        \input{src/graph/某些路径问题单log做法.tex}
        \subsection{重链剖分}
        \inputminted{cpp}{src/graph/HLD.cpp}
        \subsection{2-SAT}
        \input{src/graph/2sat.tex}
        \subsection{Dinic 网络流、费用流}
        \inputminted{cpp}{src/graph/dinic-flow.cpp}
        \inputminted{cpp}{src/graph/dinic-cost.cpp}
        \subsubsection{无源汇有上下界可行流}
        \input{src/graph/无源汇上下界可行流.tex}
        \subsubsection{有源汇有上下界最大流}
        \input{src/graph/有源汇上下界最大流.tex}
        \subsubsection{网络流总结}
        \input{src/graph/网络流总结.tex} % 从 nemesis 复制过来的,需要修改
        % 原始对偶费用流?
        \subsection{差分约束}
        \input{src/graph/差分约束.tex}
        \subsection{欧拉路径}
        \input{src/graph/euler-path.tex}
        \subsection{同余最短路}
        \input{src/graph/同余最短路.tex}
        \subsection{树的计数}
        \input{src/graph/树的计数.tex}
        % \subsection{双极定向}
        % \subsection{斯坦纳树}
        % \subsubsection{最小割树}
        \newpage

        \section{数据结构}
        \subsection{平衡树}
        \subsubsection{FHQ Treap}
        \inputminted{cpp}{src/data structure/fhq.cpp}
        \subsubsection{平衡树合并}
        \input{src/data structure/treap-merge.tex}
        \subsubsection{Splay}
        \inputminted{cpp}{src/data structure/splay.cpp}
        \subsection{LCT 动态树}
        \inputminted{cpp}{src/data structure/LCT.cpp}
        \subsection{ODT 珂朵莉树}
        \inputminted{cpp}{src/data structure/ODT.cpp}
        \subsection{李超线段树}
        \input{src/data structure/李超线段树.tex}
        \subsection{二维树状数组}
        \inputminted{cpp}{src/data structure/2D-BIT.cpp}
        \subsection{虚树}
        \input{src/data structure/虚树.tex}
        \subsection{左偏树}
        \input{src/data structure/heap.tex}
        \subsection{吉司机线段树}
        \input{src/data structure/seg-beats.tex}
        \subsection{树分治(点分治)}
        \input{src/data structure/tree-divide.tex}
        % 树分块。
        % 树状数组上二分。
        % ETT 。
        % cdq 以及其它分治。
        % 莫队。
        
        % 笛卡尔树
        \newpage

        % exchange argument 。
        
        \section{字符串}
        \subsection{Hash 类}
        \inputminted{cpp}{src/string/hash.cpp}
        \subsection{后缀数组(llx)}
        \inputminted{cpp}{src/string/llx_SA.cpp}
        \subsection{后缀数组与后缀树(旧)}
        \inputminted{cpp}{src/string/SA.cpp}
        \subsection{AC自动机}
        \inputminted{cpp}{src/string/ACAM.cpp}
        \subsection{回文自动机}
        \inputminted{cpp}{src/string/PAM.cpp}
        \subsection{Manacher算法}
        \inputminted{cpp}{src/string/manacher.cpp}
        \subsection{KMP算法与border理论}
        \input{src/string/kmp.tex}
        \subsection{Z函数}
        \input{src/string/zfunc.tex}
        \subsection{后缀自动机}
        \inputminted{cpp}{src/string/SAM.cpp}
        \subsection{后缀自动机(map版)}
        \inputminted[highlightlines={3,10,15,18}]{cpp}{src/string/SAM_map.cpp}
        \subsection{最小表示法}
        \inputminted{cpp}{src/string/min_pos.cpp}
        \newpage

        \section{线性代数}
        \subsection{高斯消元}
        \inputminted{cpp}{src/linear/gauss.cpp}
        \subsection{线性基}
        \inputminted{cpp}{src/linear/basis.cpp}
        \subsection{行列式}
        \input{src/linear/det.tex}
        \subsection{矩阵树定理}
        \input{src/linear/matrix-tree.tex}
        \subsection{单纯形法}
        \input{src/linear/simplex.tex}
        \subsection{全幺模矩阵}
        \input{src/linear/TUM.tex}
        \subsection{对偶原理}
        \input{src/linear/duality.tex}
        % 记得写格林公式之类的。
        \newpage

        \section{多项式}
        \subsection{FFT}
        \inputminted{cpp}{src/poly/fft.cpp}
        \subsection{NTT}
        \inputminted{cpp}{src/poly/ntt.cpp}
        \subsection{集合幂级数}
        \subsubsection{并卷积、交卷积与子集卷积}
        \input{src/poly/high-dim-sum.tex}
        \subsubsection{对称差卷积}
        \input{src/poly/multi-poly.tex}
        \subsection{多项式全家桶}
        \inputminted{cpp}{src/poly/poly.cpp}
        \newpage

        \section{数论}
        \subsection{类欧几里得}
        \input{src/number theory/euclid.tex}
        \subsection{中国剩余定理}
        \input{src/number theory/crt.tex}
        \subsection{扩展中国剩余定理}
        \input{src/number theory/excrt.tex}
        \subsection{BSGS}
        \input{src/number theory/bsgs.tex}
        \subsection{扩展 BSGS}
        \input{src/number theory/exbsgs.tex}
        \subsection{Lucas 定理}
        \inputminted{cpp}{src/number theory/lucas.cpp}
        \subsection{扩展 Lucas 定理}
        \inputminted{cpp}{src/number theory/exlucas.cpp}
        \subsection{杜教筛}
        \input{src/number theory/du.tex}
        \subsection{Min-25筛}
        Min-25 筛本质上是对埃氏筛进行了扩展,用于求解积性函数的前缀和,要求其在质数与质数的次幂处的取值可以被快速计算。


\subsubsection*{第一步}

求 $g(n) = \sum_{p\le n} f(p)$。


设 $g(n, j)$ = $\sum_{i=1}^n [i\text{是质数}\vee i\text{ 的最小质因子}>p_j]f(i)$。


因此,能得到递推式:
$$
g(n,j) = g(n, j-1) - f(p_j)\left(g\left(\left \lfloor  \dfrac{n}{p_j}\right \rfloor, j-1 \right) - g\left(p_{j-1}, j-1\right)\right)
$$

\subsubsection*{第二步}
\subparagraph*{做法一}


设 $S(n, j)$ = $\sum_{i=2}^n [i\text{ 的最小质因子}>p_j]f(i)$。

最后答案就是 $S(n,0) + f(1)$。

把 $S(n,j)$ 分成质数部分和合数部分进行计算。

$$
\begin{aligned}
S(n,j) &= g(n) - g(p_j) \\
 &+\sum_{j<k,p_k\le\sqrt n,p_k^{e+1}\le n }\left( f(p_k^e) S\left ( \left \lfloor \dfrac{n}{p_k^e} \right \rfloor,k  \right )  f(p_k^{e+1})\right)
\end{aligned}
$$


然后暴力递归计算。


\subparagraph*{做法二}


设 $S(n, j)$ = $\sum_{i=2}^n [i\text{是质数}\vee i\text{ 的最小质因子}>p_j]f(i)$。


写出递推式:
$$
\begin{aligned}
S(n,j) &= S(n,j+1)\\
&+\sum_{p_{j+1}\le\sqrt n,p_{j+1}^{e+1}\le n}f(p_{j+1}^e)\left(S\left ( \left \lfloor \dfrac{n}{p_{j+1}^e} \right \rfloor ,j+1 \right ) -g(p_{j+1})\right)\\
&+ f(p_{j+1}^{e+1})
\end{aligned}
$$

和第一步中一样,滚动数组 dp 即可,\textbf{边界条件} $S(x,\cdot)=g(x)$。


\textbf{注意到做法二可以求出所有 $S(\left\lfloor\frac n x\right\rfloor)$,这是做法一无法完成的。}



以下是洛谷 P5325 求积性函数 $f(p^k)=p^k(p^k-1),p\in\mathbb P$ 的前缀和的代码:
\inputminted{cpp}{src/number theory/min25.cpp}
        \subsection{没有精度问题的整除(?)}
        \inputminted{cpp}{src/number theory/div.cpp}

        % Miller–Rabin
        % Pollard Rho
        \newpage

        \section{计算几何}
        \subsection{声明与宏}
        \inputminted{cpp}{src/geometry/define.cpp}
        \subsection{点与向量}
        \inputminted{cpp}{src/geometry/vector.cpp}
        \subsection{线}
        \inputminted{cpp}{src/geometry/line.cpp}
        \subsection{圆}
        \inputminted{cpp}{src/geometry/circle.cpp}
        \subsection{凸包}
        \inputminted{cpp}{src/geometry/convex.cpp}
        \subsection{三角形}
        \inputminted{cpp}{src/geometry/triangle.cpp}
        \subsection{多边形}
        \inputminted{cpp}{src/geometry/polygon.cpp}
        \subsection{半平面交}
        \inputminted{cpp}{src/geometry/half-plane.cpp}
        \newpage

        \section{其它工具}
        \subsection{编译命令}
        \inputminted{bash}{src/tools/compile.sh}
        \subsection{快读}
        \inputminted{cpp}{src/tools/fastio.cpp}
        \subsection{Python Hints}
        \input{src/tools/python_hint.tex}
        \subsection{对拍器}
        \inputminted{python3}{src/tools/llx_checker.py}
        \subsection{常数表}
        \input{src/tools/constant_table.tex}
        \subsection{试机赛}
        \begin{itemize}
    \item 检查身份证件:护照、学生证、胸牌以及现场所需通行证。
    \item 确认什么东西能带进场。特别注意:手机、智能手表、金属(钥匙)等等。
    \item 测试鼠标、键盘、显示器和座椅。如果有问题,立刻联系工作人员。
    \item 测试编译器版本。
    \begin{itemize}
        \item  \texttt{\#include<bits/stdc++.h>}
        \item  C++20: \texttt{cin>>(s+1);}
        \item  C++17: \texttt{auto [x,y]=pair{1,"abc"};}
        \item  C++11: \texttt{auto x=1;}
    \end{itemize}
    \item 测试 \_\_int128, \_\_float128, long double
    \item 测试 pragma 是否 CE 。
    \item 测试 ‐fsanitize=address,undefined
    \item 测试本地性能与提交性能
        \begin{minted}{cpp}
        const int N=1<<20;
        for(int T=4;T;T--){
            int a[N],b[N],c[N];
            for(int i=0;i<N;i++)a[i]=i,b[i]=N-i;
            ntt_init(N);ntt(a,N,1);ntt(b,N,1);
            for(int i=0;i<N;i++)c[i]=a[i]*b[i]; // 注意不要取模谢谢
            ntt(c,N,-1);
            assert(c[0]==0xad223d);
        }
        // 本地: 291 ms
        // QOJ: 340+-40 ms
        // CF: 1000+-100 ms
        \end{minted}
    \item 测试 \verb|make| 命令。
\end{itemize}




        % \subsubsection{linux}
        % \inputminted{cpp}{src/tools/linux_checker.cpp}
        % \subsubsection{windows}
        % \inputminted{cpp}{src/tools/win_checker.cpp}
        % 构建表达式树
    \end{multicols}
\end{document}