线性规划的对偶原理:原线性规划与对偶线性规划的最优解相等。即:
$$
\begin{aligned}
    {\rm minimize}: & & c^{\rm T}x & & & & {\rm maximize}:& & b^{\rm T}y \\
    {\rm constraints}: & & Ax\ge b &  & {\rm dual} & & {\rm constraints}:& & A^{\rm T}y\le c\\
    & & x\ge 0 & & & & & & y\ge 0
\end{aligned}
$$
直观上看,对于一个最小化的线性规划,我们尝试构造一个最大化的线性规划,使得它们目标函数的最优解相同。具体的,为每个约束设置一个非负的新变量,代表其系数。对于每个原变量,其对应了一个新约束,要求原约束的线性组合的对应系数不大于原目标函数的系数,从而得到原目标函数的下界。而新目标函数则要使得原约束的组合最大化,从而得到最紧的下界。而线性规划对偶性则指出,原线性规划的最优解必然与对偶线性规划的最优解相等。

对偶线性规划具备互补松弛性。即,设 $x$ 和 $y$ 分别为原问题与对偶问题的可行解,则 $x$ 和 $y$ 均为最优解,当且仅当以下两个命题同时成立:
$$
    \begin{aligned}
        \forall j\in[1,m], x_j=0\lor\sum_{i=1}^n a_{ij}y_i=c_j\\
        \forall i\in[1,n], y_i=0\lor\sum_{j=1}^m a_{ij}x_j\le b_i\\
    \end{aligned}
$$
对偶松弛性的意义是,其指出若最优解中的变量不取 $0$ ,则对应约束在最优解中一定取等。