\documentclass[10pt, a4paper, oneside]{ctexart}
\usepackage{minted, multicol, geometry, graphicx, fancyvrb, amssymb}
\usepackage{relsize, setspace, enumitem, float, hyperref, amsmath}
\usepackage[table,xcdraw]{xcolor} % 表格背景颜色
\usepackage{comment} 

\geometry{a4paper, scale = 0.85}

\setenumerate[1]{itemsep=0pt,partopsep=0pt,parsep=\parskip,topsep=0pt}
\setitemize[1]{itemsep=0pt,partopsep=0pt,parsep=\parskip,topsep=0pt}
\setdescription{itemsep=0pt,pa
rtopsep=0pt,parsep=\parskip,topsep=0pt}
\setlength{\parindent}{0pt}
\setlength{\columnsep}{18pt}

\setminted[cpp]{
	style=xcode,
	mathescape,
	linenos,
	autogobble,
	baselinestretch=1,
	tabsize=3,
	fontsize=\scriptsize,
	%bgcolor=Gray,
	frame=single,
	framesep=1mm,
	framerule=0.3pt,
	numbersep=1mm,
	breaklines=true,
	breaksymbolsepleft=2pt,
	%breaksymbolleft=\raisebox{0.8ex}{ \small\reflectbox{\carriagereturn}}, %not moe!
	%breaksymbolright=\small\carriagereturn,
	breakbytoken=false,
	% showtabs=true,
	% tab={\relscale{0.6} $\big\vert \ \ \ $ \relscale{1}},
}

\setminted[bash]{
    style=xcode,
	mathescape,
	linenos,
	autogobble,
	baselinestretch=1,
	tabsize=3,
	fontsize=\scriptsize,
	frame=single,
	framesep=1mm,
	framerule=0.3pt,
	numbersep=1mm,
	breaklines=true,
	breaksymbolsepleft=2pt,
	breakbytoken=false,
}
\setminted[python3]{
    style=xcode,
	mathescape,
	linenos,
	autogobble,
	baselinestretch=1,
	tabsize=3,
	fontsize=\scriptsize,
	frame=single,
	framesep=1mm,
	framerule=0.3pt,
	numbersep=1mm,
	breaklines=true,
	breaksymbolsepleft=2pt,
	breakbytoken=false,
}

\title{ACM 模板}
\author{黄佳瑞,钱智煊,林乐逍}
\date{\today}

\begin{document}
    \scriptsize
    \maketitle
    \newpage
    
    \begin{multicols}{2}
        \tableofcontents
        \newpage

        \section{做题指导}
        \subsection{上机前你应该注意什么}
        \begin{enumerate}
    \item 预估你需要多少机时,以及写出来以后预计要调试多久,并在纸上记下。
    \item 先想好再上机,不要一边写一边想。
    \item 如果有好写的题,务必先写好写的。
\end{enumerate}
        \subsection{机上你应该注意什么}
        \begin{enumerate}
    \item 机时应充分利用,手速应快一些。
    \item 如果你遇到了问题(做法假了、需要分讨等),先下机并通知队友。不要占着机时想。
    \item 建议使用整块时间写题,尽量不要断断续续的写。
\end{enumerate}
        \subsection{交题前你应该注意什么}
        \begin{enumerate}
    \item \verb|long long| 开了没有。
    \item 数组开够没有。
    \item 多测清了没有。
    \item 边界数据 \verb|(corner case)| 考虑了没有
    \item 调试输出删了没有
\end{enumerate}
编译命令:
\inputminted{bash}{src/tools/compile.sh}
        \subsection{如果你的代码挂了}
        按照优先级列出:
\begin{enumerate}
    \item 如果是线段树,\textbf{认真检查}:① 每个函数里面的 pushdown ② 每个信息的 pushup
    \item 先 P 再下机。(P 还没有送来就分屏调。)
    \item 看 \verb|long long| 开了没有,数组开够没有,多测清了没有。
    \item 检查 typo ,你有没有打错一些难蚌的地方。
    \item 看看你题读错没有。
    \item 检查你的代码逻辑,即你的代码实现是否与做法一致。同时让另一个人重新读题。
    \item 怀疑做法假了。拉一个人一起看代码。
\end{enumerate}
        \subsection{另外一些注意事项}
        \begin{itemize}
    \item 千万注意节奏,不要让任何一个人在一道题上卡得太久。必要的时候可以换题。
    \item 如果你想要写题,请想好再上机。诸如分类讨论一类的题目,更要想好再上。
    \item 后期的时候可以讨论,没必要一个人挂题。
\end{itemize}
        \subsection{阴间错误集锦}
        \begin{itemize}
    \item 多测不清空。
    \item 该开 \verb|long long| 的地方不开 \verb|long long| 。一般情况下建议直接 \verb|#define int long long| 。
    \item 注意变量名打错。例如 u 打成 v 或 a 打成 t 。建议读代码的时候专门检查此类错误。
    \item 树剖,倍增LCA之类的 \verb|while| 打成 \verb|if|。
\end{itemize}
        \newpage

        \section{图论}
        \subsection{Tarjan 边双、点双(圆方树)}
        记录上一个访问的边时要记录边的编号,不能记录上一个过来的节点(因为会有重边)!!!

(如果选择在加边的时候特判,注意编号问题:用输入顺序来对应数组中位置的时候,重边跳过,但是需要 `tot+=2`。)

圆方树示意图:

\begin{figure}[H]
    \centering
    \includegraphics[width=0.6\textwidth]{src/graph/tarjan-tree.png}
    \caption{圆方树示意图}
\end{figure}

\begin{minted}{cpp}
/*** 缩点 ***/
// 找出强联通分量后,对每一条边查看是否在同一个 scc 中,如果不在就加边
void tarjan(int x)
{
	dfn[x]=low[x]=++Time,sta[++tp]=x,ins[x]=true;
	for(int i=hea[x];i;i=nex[i])
	{
		if(!dfn[ver[i]]) tarjan(ver[i]),low[x]=min(low[x],low[ver[i]]);
		else if(ins[ver[i]]) low[x]=min(low[x],dfn[ver[i]]);
	}
	if(dfn[x]==low[x])
	{
        scc++;
		do { x=sta[tp],tp--,ins[x]=false,bel[x]=scc; }
        while (dfn[x]!=low[x]);
	}
}
/*** 割点 ***/
void tarjan(int x,int Last)
// Last 是边的编号,tot 初始值为 1,i 与 i^1 互为反边
{
    dfn[x]=low[x]=++Time;
    for(int i=hea[x];i;i=nex[i])
    {
        if(!dfn[ver[i]])
        {
            tarjan(ver[i],i),low[x]=min(low[x],low[ver[i]]);
            if(low[ver[i]]>dfn[x]) edg[i]=edg[i^1]=true;
        }
        else if(dfn[ver[i]]<dfn[x] && i!=(Last^1)) low[x]=min(low[x],dfn[ver[i]]);
    }
}
/*** 圆方树、点双 ***/
// `G` 表示原图,`T` 是新建的圆方树
// 一条边也是点双
// 圆方树只存在圆 - 方边!!!
int Time,dfn[N],low[N],sta[N],siz[N];
vector<int> G[N],T[N<<1]; // 不要忘记给数组开两倍
void tarjan(int u)
{
    dfn[u]=low[u]=++Time,sta[++tp]=u;
    for(int v:G[u])
    {
        if(!dfn[v])
        {
            tarjan(v),low[u]=min(low[u],low[v]);
            // 这里对 low[v]>=dfn[u] 进行计数,根节点需要 2 个,其他节点需要 1 个,那么就是割点。
            if(low[v]==dfn[u])
            {
                int hav=0; ++All;
                for(int x=0;x!=v;tp--) x=sta[tp],T[x].pb(All),T[All].pb(x),hav++;
                T[u].pb(All),T[All].pb(u);
                siz[All]=++hav;
            }
        }
        else low[u]=min(low[u],dfn[v]);
    }
}
\end{minted}
        \subsection{O(1) LCA}
        按照顺序遍历子树,并查集更新。

\begin{minted}{cpp}
int qhea[Maxq],qver[Maxq],qnex[Maxq],qid[Maxq];
inline void addq(int x,int y,int d)
    { qver[++qtot]=y,qnex[qtot]=qhea[x],qhea[x]=qtot,qid[qtot]=d; }
void dfs(int x,int Last_node)
{
    vis[x]=true;
    for(int i=hea[x];i;i=nex[i])
        if(ver[i]!=Last_node)
            dfs(ver[i],x),fa[ver[i]]=x;
    for(int i=qhea[x];i;i=qnex[i])
        if(vis[qver[i]])
            ans[qid[i]]=Find(qver[i]);
}

for(int i=1;i<=n;i++) fa[i]=i;
for(int i=1,x,y;i<n;i++) x=rd(),y=rd(),add(x,y),add(y,x);
for(int i=1,x,y;i<=m;i++) x=rd(),y=rd(),addq(x,y,i),addq(y,x,i);
dfs(rt,rt);
for(int i=1;i<=m;i++) printf("%d\n",ans[i]);
\end{minted}
        \subsection{某些路径问题单log做法}
        路径加/单点求和\&子树求和: 设 $a[u]$ 为子树内 $d[u]$ 之和,则

\begin{enumerate}
    \item 路径加:$d[u]+w$, $d[v]+w$, $d[lca(u,v)]-w d[fa[lca(u,v)]]-w$
    \item 单点求和:$a[u]=sumd(u)$
    \item 子树求和:$suma=\sum_v d[v]*(dep[v]-dep[u]+1)$,树状数组维护2系数即可
\end{enumerate}


单点加\&子树加/路径求和: 设 $g[u]$ 表示 $1$ 到 $u$ 权值之和,则

\begin{enumerate}
    \item 单点加:子树 $d$ 加
    \item 子树加 :$u$ 每个子树内 $v$ 有 $g[v] += w*(dep[v]-dep[u]+1)$ 则维护系数加即可
    \item 路径和:$sum(u,v)=g[u]+g[v]-g[lca(u,v)]-g[fa[lca(u,v)]]$
\end{enumerate}
        \subsection{重链剖分}
        \inputminted{cpp}{src/graph/HLD.cpp}
        \subsection{2-SAT}
        2-sat 问题定义为:给定 $m$ 个布尔表达式,每个包含两个布尔变量,形如 $a\lor b$ 。求是否存在一种赋值方式,使得所有表达式都为真。

对于表达式 $a\lor b$ ,显然若 $a$ 假则 $b$ 必真,反之亦然。因此,考虑将每个变量拆为两个点,分别代表此变量取值为真或假。对于每个表达式 $a\lor b$ ,连边 $\neg a\to b$ 与 $\neg b\to a$ 。对于边 $u\to v$ ,意义为若 $u$ 为真则 $v$ 必为真。那么对于最后得到的图,如果同一变量拆出的点处在同一强连通分量中,则无解,因为在可行解中它们的取值必然是不同的,但同一强连通分量中的点取值必然相同;否则,考虑缩点以后得到的 DAG ,取其拓扑序,对于每一个变量,令其取拆出的拓扑序较大的点对应的值即可。特别的,以上的判定是必要的,但我还不知道怎么证明它是充分的。
        \subsection{Dinic 网络流、费用流}
        \inputminted{cpp}{src/graph/dinic-flow.cpp}
        \inputminted{cpp}{src/graph/dinic-cost.cpp}
        \subsubsection{无源汇有上下界可行流}
        给定无源汇流量网络 $G$,询问是否存在一种标定每条边流量的方式,使得每条边流量满足上下界同时每一个点流量平衡。

\begin{minted}{cpp}
#include <bits/stdc++.h>
using namespace std;
#define infll 0x3f3f3f3f3f3f3f3f
#define inf 0x3f3f3f3f
#define pb push_back
#define eb emplace_back
#define pa pair<int, int>
#define fi first
#define se second
typedef long long ll;
inline int rd() {
    int x = 0;
    char ch, t = 0;
    while (!isdigit(ch = getchar())) t |= ch == '-';
    while (isdigit(ch)) x = x * 10 + (ch ^ 48), ch = getchar();
    return t ? -x : x;
}
struct Dini {
#define Maxn 505
#define Maxm 20005
    int tot = 1;
    int hea[Maxn], tmphea[Maxn], dep[Maxn];
    int nex[Maxm << 1], ver[Maxm << 1], num[Maxm << 1];
    ll edg[Maxm << 1];
    inline void addedge(int x, int y, int d, int _n) {
        ver[++tot] = y, nex[tot] = hea[x], hea[x] = tot, edg[tot] = d, num[tot] = -1;
        ver[++tot] = x, nex[tot] = hea[y], hea[y] = tot, edg[tot] = 0, num[tot] = _n;
    }
    inline bool bfs(int s, int t) {
        memcpy(tmphea, hea, sizeof(hea));
        memset(dep, 0, sizeof(dep)), dep[s] = 1;
        queue<int> q;
        q.push(s);
        while (!q.empty()) {
            int cur = q.front();
            q.pop();
            if (cur == t) return true;
            for (int i = hea[cur]; i; i = nex[i])
                if (edg[i] > 0 && !dep[ver[i]])
                    dep[ver[i]] = dep[cur] + 1, q.push(ver[i]);
        }
        return false;
    }
    ll dfs(int x, int t, ll flow) {
        if (!flow || x == t) return flow;
        ll rest = flow, tmp;
        for (int i = tmphea[x]; i && rest; i = nex[i]) {
            tmphea[x] = i;
            if (dep[ver[i]] == dep[x] + 1 && edg[i] > 0) {
                if (!(tmp = dfs(ver[i], t, min(rest, edg[i])))) dep[ver[i]] = 0;
                edg[i] -= tmp, edg[i ^ 1] += tmp, rest -= tmp;
            }
        }
        return flow - rest;
    }
    inline ll solve(int s, int t) {
        ll sum = 0;
        while (bfs(s, t))
            sum += dfs(s, t, infll);
        return sum;
    }
#undef Maxn
#undef Maxm
} G;
#define Maxn 205
#define Maxm 10205
int n, m, ss, tt;
int ans[Maxm];
ll needin, needout;
ll Ind[Maxn], Outd[Maxn];
int main() {
    n = rd(), m = rd(), ss = n + 1, tt = n + 2;
    for (int i = 1, x, y, Inf, Sup; i <= m; i++) {
        x = rd(), y = rd(), Inf = rd(), Sup = rd();
        G.addedge(x, y, Sup - Inf, i);
        Outd[x] += Inf;
        Ind[y] += Inf;
        ans[i] = Inf;
    }
    for (int i = 1; i <= n; i++) {
        if (Ind[i] > Outd[i])
            G.addedge(ss, i, Ind[i] - Outd[i], -1), needin += Ind[i] - Outd[i];
        if (Ind[i] < Outd[i])
            G.addedge(i, tt, Outd[i] - Ind[i], -1), needout += Outd[i] - Ind[i];
    }
    ll tmp = G.solve(ss, tt);
    if (needin != needout || needin != tmp) printf("NO\n");
    else {
        for (int i = 2; i <= G.tot; i++)
            if (G.num[i] != -1) ans[G.num[i]] += G.edg[i];
        printf("YES\n");
        for (int i = 1; i <= m; i++) printf("%d\n", ans[i]);
    }
    return 0;
}
\end{minted}
        \subsubsection{有源汇有上下界最大流}
        给定有源汇流量网络 $G$,询问是否存在一种标定每条边流量的方式,使得每条边流量满足上下界同时除了源点和汇点每一个点流量平衡。

假设源点为 $S$,汇点为 $T$,则我们可以加入一条 $T$ 到 $S$ 的上界为 $\infty$,下界为 $0$ 的边转化为无源汇上下界可行流问题。

若有解,则 $S$ 到 $T$ 的可行流流量等于 $T$ 到 $S$ 的附加边的流量。

\begin{minted}{cpp}
#include <bits/stdc++.h>
using namespace std;
#define infll 0x3f3f3f3f3f3f3f3f
#define inf 0x3f3f3f3f
#define pb push_back
#define eb emplace_back
#define pa pair<int, int>
#define fi first
#define se second
typedef long long ll;
inline int rd() {
    int x = 0;
    char ch, t = 0;
    while (!isdigit(ch = getchar())) t |= ch == '-';
    while (isdigit(ch)) x = x * 10 + (ch ^ 48), ch = getchar();
    return t ? -x : x;
}
struct Dinic {
#define Maxn 505
#define Maxm 20005
    int tot = 1;
    int hea[Maxn], tmphea[Maxn], dep[Maxn];
    int nex[Maxm << 1], ver[Maxm << 1], num[Maxm << 1];
    ll edg[Maxm << 1];
    inline int addedge(int x, int y, ll d, int _n) {
        ver[++tot] = y, nex[tot] = hea[x], hea[x] = tot, edg[tot] = d, num[tot] = -1;
        ver[++tot] = x, nex[tot] = hea[y], hea[y] = tot, edg[tot] = 0, num[tot] = _n;
        return tot;
    }
    inline bool bfs(int s, int t) {
        memcpy(tmphea, hea, sizeof(hea));
        memset(dep, 0, sizeof(dep)), dep[s] = 1;
        queue<int> q;
        q.push(s);
        while (!q.empty()) {
            int cur = q.front();
            q.pop();
            if (cur == t) return true;
            for (int i = hea[cur]; i; i = nex[i])
                if (edg[i] > 0 && !dep[ver[i]])
                    dep[ver[i]] = dep[cur] + 1, q.push(ver[i]);
        }
        return false;
    }
    ll dfs(int x, int t, ll flow) {
        if (!flow || x == t) return flow;
        ll rest = flow, tmp;
        for (int i = tmphea[x]; i && rest; i = nex[i]) {
            tmphea[x] = i;
            if (dep[ver[i]] == dep[x] + 1 && edg[i] > 0) {
                if (!(tmp = dfs(ver[i], t, min(rest, edg[i]))))
                    dep[ver[i]] = 0;
                edg[i] -= tmp, edg[i ^ 1] += tmp, rest -= tmp;
            }
        }
        return flow - rest;
    }
    inline ll solve(int s, int t) {
        ll sum = 0;
        while (bfs(s, t))
            sum += dfs(s, t, infll);
        return sum;
    }
#undef Maxn
#undef Maxm
} G;
#define Maxn 505
#define Maxm 20005
int n, m, ss, tt, s, t;
int ans[Maxm];
ll needin, needout;
ll Ind[Maxn], Outd[Maxn];
int main() {
    n = rd(), m = rd(), s = rd(), t = rd(), ss = n + 1, tt = n + 2;
    for (int i = 1, x, y, Inf, Sup; i <= m; i++) {
        x = rd(), y = rd(), Inf = rd(), Sup = rd();
        G.addedge(x, y, Sup - Inf, i);
        Outd[x] += Inf;
        Ind[y] += Inf;
        ans[i] = Inf;
    }
    for (int i = 1; i <= n; i++) {
        if (Ind[i] > Outd[i])
            G.addedge(ss, i, Ind[i] - Outd[i], -1), needin += Ind[i] - Outd[i];
        if (Ind[i] < Outd[i])
            G.addedge(i, tt, Outd[i] - Ind[i], -1), needout += Outd[i] - Ind[i];
    }
    G.addedge(t, s, infll, -1);
    ll tmp = G.solve(ss, tt);
    if (needin != needout || needin != tmp) printf("please go home to sleep\n");
    else {
        tmp = G.edg[G.tot];
        G.edg[G.tot] = G.edg[G.tot - 1] = 0;
        tmp += G.solve(s, t);
        printf("%lld\n", tmp);
    }
    return 0;
}
\end{minted}
        \subsubsection{网络流总结}
        \input{src/graph/网络流总结.tex} % 从 nemesis 复制过来的,需要修改
        % 原始对偶费用流?
        \subsection{差分约束}
        \textbf{差分约束系统} 是一种特殊的 $n$ 元一次不等式组 。

差分约束系统中的每个约束条件 $x_i-x_j\le c_k$ 都可以变形成 $x_i\le x_j+c_k$ 与 $x_j\ge x_i-c_k$ ,这与单源最短路中的三角形不等式非常相似。因此,我们可以把每个变量 $x_i$ 看做图中的一个结点,对于每个约束条件连边。

需要注意的是,有些题目看能会对解的上、下界进行约束,因此我们需要对这些条件处理(这里只考虑对于这 $n$ 个元素 \textbf{只约束了上界} 或 \textbf{只约束了下界} ):

\begin{itemize}
    \item 只约束下界:有 $0$ 号点向每一个点连一条长为 $Lim_i$ 的边,表示第 $i$ 号元素的 \textbf{下界} 为 $Lim_i$ ,如图所示建边:
    \begin{center}
        \begin{tabular}{|c|c|c|}
            \hline
            题意 & 转化 & 连边 \\
            \hline
            $x_a-x_b\ge c$ & $x_a\ge x_b+c$ & \texttt{add(b,a,c)} \\
            $x_a-x_b\le c$ & $x_b\ge x_a-c$ & \texttt{add(a,b,-c)} \\
            $x_a=x_b$ & $x_a\ge x_b$ , $x_b\le x_a$ & \texttt{add(a,b,0),add(b,a,0)} \\
            \hline
        \end{tabular}
    \end{center}
    之后对整张图跑 \textbf{最长路} 。
    \item 只约束上界:有 $0$ 号点向每一个点连一条长为 $Lim_i$ 的边,表示第 $i$ 号元素的 \textbf{上界} 为 $Lim_i$ ,如图所示建边:
    \begin{center}
        \begin{tabular}{|c|c|c|}
            \hline
            题意 & 转化 & 连边 \\
            \hline
            $x_a-x_b\ge c$ & $x_b\le x_a-c$ & \texttt{add(a,b,-c)} \\
            $x_a-x_b\le c$ & $x_a\le x_b+c$ & \texttt{add(b,a,c)} \\
            $x_a=x_b$ & $x_a\le x_b$ , $x_b\le x_a$ & \texttt{add(a,b,0),add(b,a,0)} \\
            \hline
        \end{tabular}
    \end{center}
    之后对整张图跑 \textbf{最短路} 。
\end{itemize}

设 $dist[0]=0$ ,若存在负环 $/$ 正环,则不等式无解,否则 $x_i=dist[i]$ 是该差分约束系统的一组解 。

最坏情况下(存在负环 $/$ 正环)复杂度为 $O(nm)$ 。

\textbf{注意:整个图不一定是联通的!} 
        \subsection{欧拉路径}
        \subsubsection{有向图}
\begin{minted}[highlightlines={7,9}]{cpp}
vector<int> G[N];
int cur[N];
void dfs(int u){
    for(int& i=cur[u];i<G[u].size();){ 
        i++; // 提前加 i
        dfs(G[u][i-1]); 
        // 在这里往栈中加入边 (u -> G[u][i-1])
    }
    // 在这里往栈中加入点 u
    ans.push(u);
}    
\end{minted}
\subsubsection{无向图}
\begin{minted}[highlightlines={19,21}]{cpp}
struct Edge{
    int to,rev; // 终点,反向边编号
    bool exist;
};
vector<Edge> G[N];
int cur[N];
stack<int> ans;
void add_edge(int u,int v){
    G[u].push_back(Edge{v,(int)G[v].size(),true});
    if(u!=v)G[v].push_back(Edge{u,(int)G[u].size()-1,true});
}
void dfs(int u){
    for(int& i=cur[u];i<G[u].size();){
        i++;
        if(!G[u][i-1].exist)continue;
        G[u][i-1].exist = 0;
        G[G[u][i-1].to][G[u][i-1].rev].exist=0;
        dfs(G[u][i-1].to);
        // 在这里加入边 (u->G[u][i-1].to)
    }
    // 在这里加入节点 u
}
\end{minted}
        \subsection{同余最短路}
        形如:
\begin{itemize}
    \item 设问 $1$ :给定 $n$ 个整数,求这 $n$ 个整数在 $h(h\le2^{63}-1)$ 范围内 \textbf{能拼凑出多少的其他整数(整数可以重复取)} 。
    \item 设问 $2$ :给定 $n$ 个整数,求这 $n$ 个整数 \textbf{不能拼凑出的最小(最大)的整数} 。
\end{itemize}

设 $x$ 为 $n$ 个数中最小的一个,令 $ds[i]$ 为只通过增加其他 $n-1$ 种数能够达到的最低楼层 $p$ ,并且满足 $p\equiv i\pmod{x}$ 。

对于 $n-1$ 个数与 $x$ 个 $ds[i]$ ,可以如下连边:

\begin{minted}{cpp}
for(int i=0;i<x;i++) for(int j=2;j<=n;j++) add(i,(i+a[j])%x,a[j]);
\end{minted}

之后进行最短路 ,对于 :
\begin{itemize}
    \item 问在 $h$ 范围内能够到达的\textbf{点的数量}:答案为(加一因为 $i$ 本身也要计算)
    \[
    \sum_{i=0}^{x-1}{[d[i]\le h]\times\dfrac{h-d[i]}{x}+1}
    \]
    \item 问不能达到的\textbf{最小的数}:答案为:( $i$ 一定时最小表示的数为 $d[i]=s\times x+i$ ,则 $(s-1)\times x+i$  \textbf{一定不能} 被表示出来 )
    \[
    \min_{i=1}^{x-1}{\{d[i]-x\}}
    \]
\end{itemize}

\textbf{注意:$ds$ 与 $h$ 范围相同,一般也要开 $\operatorname{long}~\operatorname{long}$ !}
        % \subsection{双极定向}
        % \subsection{斯坦纳树}
        % \subsubsection{最小割树}
        \newpage

        \section{数据结构}
        \subsection{平衡树}
        \subsubsection{FHQ Treap}
        \inputminted{cpp}{src/data structure/fhq.cpp}
        \subsubsection{平衡树合并}
        如果需要合并两个有交集的 Treap 时该怎么做?我们可以每次将较小的数合并到较大的树中去,这样每个点最多只会合并 $\log n$ 次,每次合并复杂度 $O(n\log n)$,总时间复杂度 $O(n\log n\log V)$。

代码其实非常暴力,就是直接对更小的那棵树直接一个个插入进去:

\inputminted{cpp}{src/data structure/treap-merge.cpp}

\href{https://codeforces.com/blog/entry/108601}{可以证明},若只支持合并与分裂操作,则时间复杂度为 $O(n\log n)$ 。
        \subsubsection{Splay}
        \inputminted{cpp}{src/data structure/splay.cpp}
        \subsection{LCT 动态树}
        \inputminted{cpp}{src/data structure/LCT.cpp}
        \subsection{ODT 珂朵莉树}
        \inputminted{cpp}{src/data structure/ODT.cpp}
        \subsection{李超线段树}
        \input{src/data structure/李超线段树.tex}
        \subsection{二维树状数组}
        \inputminted{cpp}{src/data structure/2D-BIT.cpp}
        \subsection{虚树}
        $\bigstar\texttt{attention}$:由于整个图需要用到的边与点很少,所以在每次新建虚树的时候不能全局清空,而是在把一个新的点加入栈中的时候清空这个点连过的边。

$\bigstar\texttt{attention}$:一定要在将这个点加入栈的时候清空这个点的连边情况!!!!不能早也不能晚。

\begin{minted}{cpp}
inline void build(int s)
{
    sort(dot+1,dot+m+1,cmp),tot=0;
    sta[tp=1]=1,hea[1]=0;
    for(int i=1,l;i<=m;i++)
    {
        if(dot[i]==1) continue; // 别忘了 1! 
        l=Lca(sta[tp],dot[i]);
        if(l!=sta[tp])
        {
            while(dfn[l]<dfn[sta[tp-1]])
                add(sta[tp-1],sta[tp],dist(sta[tp]-sta[tp-1])), tp--;
            if(sta[tp-1]==l)
                add(l,sta[tp],dist(sta[tp]-l)),tp--;
            else
                hea[l]=0, add(l,sta[tp],dist(sta[tp]-l)), sta[tp]=l;
        }
        hea[dot[i]]=0,sta[++tp]=dot[i];
    }
    for(int i=1;i<tp;i++) add(sta[i],sta[i+1],dist(sta[i+1]-sta[i]));
}
\end{minted}
        \subsection{左偏树}
        \inputminted{cpp}{src/data structure/heap.cpp}
特别的,若要求给定节点所在左偏树的根,须使用并查集。对于每个节点维护 rt[] 值,查找根时使用函数:
\begin{minted}{cpp}
int find(int x) { return rt[x] == x ? x : rt[x] = find(rt[x]); }
\end{minted}
在合并节点时,加入:
\begin{minted}{cpp}
rt[x] = rt[y] = merge(x, y);
\end{minted}
在弹出最小值时加入:
\begin{minted}{cpp}
rt[ls(x)] = rt[rs(x)] = rt[x] = merge(ls(x), rs(x));
\end{minted}
另外,删除过的点是不能复用的,因为这些点可能作为并查集的中转节点。
        \subsection{吉司机线段树}
        \begin{itemize}
    \item 区间取 min 操作:通过维护区间次小值实现,即将区间取 min 转化为对区间最大值的加法,当要取 min 的值 v 大于次小值时停止递归。时间复杂度通过标记回收证明,即将区间最值视作标记,这样每次多余的递归等价于标记回收,总时间复杂度为 $O(m\log n)$。
    \item 区间历史最大值:通过维护加法标记的历史最大值实现。
\end{itemize}
\inputminted{cpp}{src/data structure/seg-beats.cpp}
        \subsection{树分治(点分治)}
        \begin{minted}{cpp}
/*** (静态)点分治 ***/
void Find1(int x,int fa)
{
    siz[x]=1,subsiz++;
    for(int i=hea[x];i;i=nex[i]) if(!used[ver[i]] && ver[i]!=fa)
        Find1(ver[i],x),siz[x]+=siz[ver[i]];
}
void Find2(int x,int fa)
{
    bool isrt=true;
    for(int i=hea[x];i;i=nex[i]) if(!used[ver[i]] && ver[i]!=fa)
    {
        Find2(ver[i],x);
        if((siz[ver[i]]<<1)>subsiz) isrt=false;
    }
    if(((subsiz-siz[x])<<1)>subsiz) isrt=false;
    if(isrt) rt=x;
}
void solve(int x)
{
    subsiz=0,Find1(x,0),Find2(x,0);
    // 处理和 rt 有关的答案
    used[rt]=true;
    for(int i=hea[rt];i;i=nex[i]) if(!used[ver[i]]) solve(ver[i]);
}
\end{minted}

熟知序列分治的过程是选取恰当的分治点并考虑所有跨过分治点的区间。而树分治的过程也是类似的,以点分治为例,每一次选择当前联通块的重心作为分治点,然后考虑所有跨越分治点的路径,并对分割出的联通块递归。

若要处理树上邻域问题,可以考虑建出点分树。处理点 x 的询问时,只需考虑 x 在点分树上到根的路径,每一次加上除开 x 所在子树的答案即可。

\inputminted{cpp}{src/data structure/tree-divide.cpp}
        \subsection{树的计数}
        \subsubsection{树的计数 Prufer序列}
    prufer编码长度为${n-2}$, 且度数为$d_i$ 的点在prufer编码中出现${d_i -1}$次. 
    \par 由树得到序列: 总共需要$n-2$步, 第$i$步在当前的树中寻找具有最小标号的叶子节点, 将与其相连的点的标号设为Prufer序列的第$i$个元素$p_i$, 并将此叶子节点从树中删除, 直到最后得到一个长度为$n-2$的Prufer 序列和一个只有两个节点的树. 
    \par 由序列得到树: 先将所有点的度赋初值为$1$, 然后加上它的编号在Prufer序列中出现的次数, 得到每个点的度; 执行$n-2$步, 第$i$步选取具有最小标号的度为$1$的点$u$与$v=p_i$ 相连, 得到树中的一条边, 并将$u$和$v$ 的度减一. 最后再把剩下的两个度为$1$的点连边, 加入到树中. 
    \par 相关结论: $n$个点完全图, 每个点度数依次为$d_1$,$d_2$,...,$d_n$, 这样生成树的棵树为: ${\frac{(n-2)!}{(d_1-1)!(d_2-1)!...(d_n-1)!}}$.\\
    左边有$n_1$个点, 右边有$n_2$个点的完全二分图的生成树棵树为$n_1^{n_2-1}\times n_2^{n_1-1}$. \\
    $m$个连通块, 每个连通块有$c_i$个点, 把他们全部连通的生成树方案数: $(\sum c_i)^{m-2}\prod c_i$
\subsubsection{有根树计数 1,1,2,4,9,20,48,115,286,719,1842,4766}\noindent
无标号 $a_{n+1} = 1/n  \sum_{k=1}^{n} ( \sum_{d|k} d \cdot a(d) ) \cdot a(n-k+1)$ 
\subsubsection{无根树计数}\noindent
    $n$是奇数时, 有$a_n-\sum_{i}^{n/2}a_ia_{n-i}$种不同的无根树. \\
    $n$时偶数时, 有$a_n-\sum_{i}^{n/2}a_ia_{n-i}+\frac{1}{2}a_{n/2}(a_{n/2}+1)$种不同的无根树. 
\subsubsection{生成树计数 Kirchhoff's Matrix-Tree Thoerem}
    Kirchhoff Matrix $T=Deg-A$, $Deg$是度数对角阵, $A$是邻接矩阵. 无向图度数矩阵是每个点度数; 有向图度数矩阵是每个点入度.\\
    邻接矩阵$A[u][v]$表示$u\rightarrow v$边个数, 重边按照边数计算, 自环不计入度数.\\
    无向图生成树计数: $c=|K$的任意1个$n-1$阶主子式$|$\\
    有向图外向树计数: $c=|$去掉根所在的那阶得到的主子式$|$
\subsubsection{有向图欧拉回路计数 BEST Thoerem}
        \[ \mathrm{ec}(G) = t_w(G)\prod_{v \in{V}}(\mathrm{deg}(v) - 1)! \]
        其中$\mathrm{deg}$为入度(欧拉图中等于出度), $t_w(G)$为以$w$为根的外向树的个数. 相关计算参考生成树计数.\\
        欧拉连通图中任意两点外向树个数相同: $\mathrm{t_v}(G) = \mathrm{t_w}(G)$.
        % 树分块。
        % 树状数组上二分。
        % ETT 。
        % cdq 以及其它分治。
        % 莫队。
        \newpage

        % exchange argument 。
        
        \section{字符串}
        \subsection{Hash 类}
        \inputminted{cpp}{src/string/hash.cpp}
        \subsection{后缀数组(llx)}
        \inputminted{cpp}{src/string/llx_SA.cpp}
        \subsection{后缀数组与后缀树(旧)}
        \inputminted{cpp}{src/string/SA.cpp}
        \subsection{AC自动机}
        \inputminted{cpp}{src/string/ACAM.cpp}
        \subsection{回文自动机}
        \inputminted{cpp}{src/string/PAM.cpp}
        \subsection{Manacher算法}
        \inputminted{cpp}{src/string/manacher.cpp}
        \subsection{KMP算法与border理论}
        \inputminted{cpp}{src/string/kmp.cpp}
字符串的border理论:
以下记字符串 $S$ 的长度为 $n$ 。
\begin{itemize}
    \item 若串 $S$ 具备长度为 $m$ 的 border ,则其必然具备长度为 $n-m$ 的周期,反之亦然。
    \item 弱周期性引理:若串 $S$ 存在周期 $p$ 、$q$ ,且 $p+q\le n$ ,则 $S$ 必然存在周期 $\gcd(p,q)$ 。
    \item 引理1:若串 $S$ 存在长度为 $m$ 的 border $T$,且 $T$ 具备周期 $p$ ,满足 $2m-n\ge p$ ,则 $S$ 同样具备周期 $p$ 。
    \item 周期性引理:若串 $S$ 存在周期 $p$ 、$q$ ,满足 $p+q-\gcd(p,q)\le n$ ,则串 $S$ 必然存在周期 $\gcd(p,q)$ 。
    \item 引理2:串 $S$ 的所有 border 的长度构成了 $O(\log n)$ 个不交的等差数列。更具体的,记串 $S$ 的最小周期为 $p$ ,则其所有长度包含于区间 $[n \bmod p + p, n)$ 的 border 构成了一个等差数列。
    \item 引理3:若存在串 $S$ 、$T$ ,使得 $2|T|\ge n$ ,则 $T$ 在 $S$ 中的所有匹配位置构成了一个等差数列。
    \item 引理4:PAM 的失配链可以被划分为 $O(\log n)$ 个等差数列。
\end{itemize}
        \subsection{Z函数}
        Z函数用于求解字符串的每一个后缀与其本身的 lcp 。其思路和 manacher 算法基本一致,都是维护一个扩展过的最右端点和对应的起点,而当前点要么暴力扩展使最右端点右移,要么处在记录的起点和终点间,从而可以利用已有的信息快速转移。
\inputminted{cpp}{src/string/zfunc.cpp}
        \subsection{后缀自动机}
        \inputminted{cpp}{src/string/SAM.cpp}
        \subsection{后缀自动机(map版)}
        \inputminted[highlightlines={3,10,15,18}]{cpp}{src/string/SAM_map.cpp}
        \subsection{最小表示法}
        \inputminted{cpp}{src/string/min_pos.cpp}
        \newpage

        \section{线性代数}
        \subsection{高斯消元}
        \inputminted{cpp}{src/linear/gauss.cpp}
        \subsection{线性基}
        \inputminted{cpp}{src/linear/basis.cpp}
        \subsection{行列式}
        容易证明转置后行列式相等,积的行列式等于行列式的积。但是这样的性质并不对和成立,而每次只能拆一行或一列。

以下是任意模数求行列式的算法:
\inputminted{cpp}{src/linear/det.cpp}
        \subsection{矩阵树定理}
        以下叙述允许重边,不允许自环。

对于无向图 $G$ ,定义度数矩阵 $D$ 为:
$$
    D_{ij} = \deg(i)[i=j]
$$
设 $\#e(i,j)$ 为连接点 $i$ 和 $j$ 的边数,定义邻接矩阵 $A$ 为:
$$
    A_{ij} = \#e(i,j)
$$
显然 $A_{ii}=0$ 。
定义 Laplace 矩阵 $L$ 为 $D-A$ ,记 $G$ 的生成树个数为 $t(G)$ ,则其恰为 $L$ 的任意一个 $n-1$ 阶主子式的值。

对于有向图 $G$ ,分别定义出度矩阵 $D^{out}$ 和入度矩阵 $D^{in}$ 为:
$$
    \begin{aligned}
        D^{out}_{ij} & = \deg^{out}(i)[i=j] \\
        D^{in}_{ij}  & = \deg^{in}(i)[i=j]
    \end{aligned}
$$
设 $\#e(i,j)$ 为从点 $i$ 到 $j$ 的边数,定义邻接矩阵 $A$ 为:
$$
    A_{ij} = \#e(i,j)
$$
显然 $A_{ii}=0$ 。
再分别定义出度 Laplace 矩阵 $L^{out}$ 和入度 Laplace 矩阵 $L^{in}$ 为:
$$
    \begin{aligned}
        L^{out} & = D^{out}-A \\
        L^{in}  & = D^{in}-A
    \end{aligned}
$$
分别记 $G$ 的以 $k$ 为根的根向树形图个数为 $t^{root}(k)$ ,以及以 $k$ 为根的叶向树形图个数为 $t^{leaf}(k)$ 。则 $t^{root}(k)$ 恰为 $L^{out}$ 的删去 $k$ 行 $k$ 列的 $n-1$ 阶主子式的值;$t^{leaf}(k)$ 恰为 $L^{in}$ 的删去 $k$ 行 $k$ 列的 $n-1$ 阶主子式的值。
        \subsection{单纯形法}
        线性规划的标准型:

$$
\begin{aligned}
	{\rm maximize}: & & c^{\rm T}x\\
	{\rm constraints}: & & Ax\le b\\
	& & x\ge 0
\end{aligned}
$$

在标准型的基础上得到松弛型:

$$
\begin{aligned}
    {\rm maximize}: & & c^{\rm T}x\\
    {\rm constraints}: & & \alpha=b-Ax\\
    & & \alpha,x\ge 0
\end{aligned}
$$

单纯性法以松弛型为基础。具体的,松弛型隐含了一个基本解,即 $x=0$ ,$\alpha = b$(这里要求 $b\ge 0$ )。我们称 $\alpha$ 中的变量为基变量,其余为非基变量。单纯性的主过程被称作 pivot 操作。一次 pivot 操作的本质就是进行基变量与非基变量之间的变换以使得带入基本解的目标函数更大。具体的,我们每一次选定一个在目标函数中系数为正的变量为换入变量,再选择对这个换入变量约束最紧的的线性约束所对应的基变量,称其为换出变量。然后,我们将换入变量和换出变量分别换为基变量和非基变量,并对其余的式子做出对应的代换以使得定义满足即可。

另外单纯型法的时间复杂度虽然是指数级别的,但是跑起来效果还是很好的,期望迭代次数貌似可以大致看作约束个数的平方级别。

\inputminted{cpp}{src/linear/simplex.cpp}
        \subsection{全幺模矩阵}
        当一个矩阵的任意一个子方阵的行列式都为 $\pm1,0$ 时,我们称这个矩阵是全幺模的。

如果单纯形矩阵是全幺模的,那么单纯形就具有整数解。
        \subsection{对偶原理}
        线性规划的对偶原理:原线性规划与对偶线性规划的最优解相等。即:
$$
\begin{aligned}
    {\rm minimize}: & & c^{\rm T}x & & & & {\rm maximize}:& & b^{\rm T}y \\
    {\rm constraints}: & & Ax\ge b &  & {\rm dual} & & {\rm constraints}:& & A^{\rm T}y\le c\\
    & & x\ge 0 & & & & & & y\ge 0
\end{aligned}
$$
直观上看,对于一个最小化的线性规划,我们尝试构造一个最大化的线性规划,使得它们目标函数的最优解相同。具体的,为每个约束设置一个非负的新变量,代表其系数。对于每个原变量,其对应了一个新约束,要求原约束的线性组合的对应系数不大于原目标函数的系数,从而得到原目标函数的下界。而新目标函数则要使得原约束的组合最大化,从而得到最紧的下界。而线性规划对偶性则指出,原线性规划的最优解必然与对偶线性规划的最优解相等。

对偶线性规划具备互补松弛性。即,设 $x$ 和 $y$ 分别为原问题与对偶问题的可行解,则 $x$ 和 $y$ 均为最优解,当且仅当以下两个命题同时成立:
$$
    \begin{aligned}
        \forall j\in[1,m], x_j=0\lor\sum_{i=1}^n a_{ij}y_i=c_j\\
        \forall i\in[1,n], y_i=0\lor\sum_{j=1}^m a_{ij}x_j\le b_i\\
    \end{aligned}
$$
对偶松弛性的意义是,其指出若最优解中的变量不取 $0$ ,则对应约束在最优解中一定取等。
        % 记得写格林公式之类的。
        \newpage

        \section{多项式}
        \subsection{FFT}
        \inputminted{cpp}{src/poly/fft.cpp}
        \subsection{NTT}
        \inputminted{cpp}{src/poly/ntt.cpp}
        \subsection{集合幂级数}
        \subsubsection{并卷积、交卷积与子集卷积}
        集合并等价于二进制按位或,因此并卷积的计算实际上就是做高维前缀和以及差分,也被称作莫比乌斯变换。
\inputminted{cpp}{src/poly/or.cpp}
而集合交卷积则对应后缀和。
\inputminted{cpp}{src/poly/and.cpp}
子集卷积则较为特殊,为了使得产生贡献的集合没有交集,考虑引入代表集合大小的占位符。这样只需做 $n$ 次 FMT ,再枚举长度做 $n^2$ 次卷积。因为 FMT 具备线性性,所以最后只需做 $n$ 次 iFMT 即可。
\inputminted{cpp}{src/poly/linear.cpp}
特别的,子集卷积等价于 $n$ 元保留到一次项的线性卷积。
        \subsubsection{对称差卷积}
        集合对称差等价于按位异或,而异或卷积则等价于 $n$ 元模 $2$ 的循环卷积,因此,FWT 实质上和 $n$ 元 FFT 没有什么区别。
\inputminted{cpp}{src/poly/xor.cpp}
        \subsection{多项式全家桶}
        \inputminted{cpp}{src/poly/poly.cpp}
        \newpage

        \section{数论}
        \subsection{中国剩余定理}
        $$
\begin{cases}
	x\equiv a_1 \pmod {m_1}\\
	x\equiv a_2 \pmod {m_2}\\
	\cdots\\
	x\equiv a_n \pmod {m_n}\\
\end{cases}
$$
求解 $x$ ,其中 $m_1,m_2,\dots m_n$ 互素。
$$
x\equiv\sum_{i=1}^na_i\prod_{j\neq i}^n m_j\times\left(\left(\prod_{j\neq i}^n m_j\right)^{-1}\pmod{m_i}\right)\pmod{\prod_{i=1}^n m_i}
$$
        \subsection{扩展中国剩余定理}
        用于求解同余方程组的模数并不互素的情况。我们考虑如何合并两个同余式:
$$
\begin{cases}
	x \equiv a_1 \pmod {m_1}\\
	x \equiv a_2 \pmod {m_2}\\
\end{cases}
$$
显然其等价于:
$$
\begin{cases}
	x = a_1 + k_1\times m_1\\
	x = a_2 + k_2\times m_2\\
\end{cases}
$$
联立可得:
$$
k_1m_1-k_2m_2=a_2-a_1
$$
我们解这个方程即可得出当前的解 $x_0$ 。且注意到我们若给 $x_0$ 加上若干个 $\operatorname{lcm}(m_1,m_2)$ ,上式仍然成立,即当前的解是在 $\bmod \operatorname{lcm}(m_1,m_2)$ 意义下的。这样我们得出新的同余式:
$$
x\equiv x_0 \pmod {\operatorname{lcm}(m_1,m_2)}
$$
与其它式子继续合并即可。注意在数据范围比较大的时候需要龟速加。
        \subsection{BSGS}
        在 $\sqrt{p}$ 的时间内求解 $a^x\equiv b\pmod{p}$,要求 $a$ 与 $p$ 互质。
\inputminted{cpp}{src/number theory/bsgs.cpp}
        \subsection{扩展 BSGS}
        不要求 $a,p$ 互质。
\inputminted{cpp}{src/number theory/exbsgs.cpp}
        \subsection{Lucas 定理}
        \inputminted{cpp}{src/number theory/lucas.cpp}
        \subsection{扩展 Lucas 定理}
        \inputminted{cpp}{src/number theory/exlucas.cpp}
        \subsection{杜教筛}
        实际上是利用迪利克雷卷积来构造递推式,从而对一些积性函数快速求和的方法。

我们现在考虑求取积性函数 $f$ 的前缀和 $F$ 。设存在函数 $g$ ,使得 $f*g$ 的前缀和可以被快速计算,那么:
$$
\begin{aligned}
	\sum_{k=1}^n(f*g)(k)
	&=\sum_{k=1}^n\sum_{d\mid k}f\left(\frac kd\right)\times g(d)\\
	&=\sum_{d=1}^n\sum_{k=1}^{\lfloor n/ d \rfloor} f(k) \times g(d)\\
	&=\sum_{d=1}^ng(d)\times F\left(\left\lfloor\frac nd\right\rfloor\right)\\
	&=\sum_{d=2}^ng(d)\times F\left(\left\lfloor\frac nd\right\rfloor\right)+g(1)\times F(n)\\
\end{aligned}
$$
则:
$$
F(n)=\left(\sum_{k=1}^n(f*g)(k)-\sum_{d=2}^ng(d)\times F\left(\left\lfloor\frac nd\right\rfloor\right)\right){\bigg /} g(1)
$$
若 $f*g$ 的前缀和可以被快速计算,我们就可以使用整除分块,从而把 $F(n)$ 划分为若干个子问题。使用时使用线性筛来预处理 $F$ 的前 $n^{\frac 23}$ 项,这样杜教筛的时间复杂度为 $O(n^{\frac 23})$ 。
        \subsection{Min-25筛}
        Min-25 筛本质上是对埃氏筛进行了扩展,用于求解积性函数的前缀和,要求其在质数与质数的次幂处的取值可以被快速计算。


\subsubsection*{第一步}

求 $g(n) = \sum_{p\le n} f(p)$。


设 $g(n, j)$ = $\sum_{i=1}^n [i\text{是质数}\vee i\text{ 的最小质因子}>p_j]f(i)$。


因此,能得到递推式:
$$
g(n,j) = g(n, j-1) - f(p_j)\left(g\left(\left \lfloor  \dfrac{n}{p_j}\right \rfloor, j-1 \right) - g\left(p_{j-1}, j-1\right)\right)
$$

\subsubsection*{第二步}
\subparagraph*{做法一}


设 $S(n, j)$ = $\sum_{i=2}^n [i\text{ 的最小质因子}>p_j]f(i)$。

最后答案就是 $S(n,0) + f(1)$。

把 $S(n,j)$ 分成质数部分和合数部分进行计算。

$$
\begin{aligned}
S(n,j) &= g(n) - g(p_j) \\
 &+\sum_{j<k,p_k\le\sqrt n,p_k^{e+1}\le n }\left( f(p_k^e) S\left ( \left \lfloor \dfrac{n}{p_k^e} \right \rfloor,k  \right )  f(p_k^{e+1})\right)
\end{aligned}
$$


然后暴力递归计算。


\subparagraph*{做法二}


设 $S(n, j)$ = $\sum_{i=2}^n [i\text{是质数}\vee i\text{ 的最小质因子}>p_j]f(i)$。


写出递推式:
$$
\begin{aligned}
S(n,j) &= S(n,j+1)\\
&+\sum_{p_{j+1}\le\sqrt n,p_{j+1}^{e+1}\le n}f(p_{j+1}^e)\left(S\left ( \left \lfloor \dfrac{n}{p_{j+1}^e} \right \rfloor ,j+1 \right ) -g(p_{j+1})\right)\\
&+ f(p_{j+1}^{e+1})
\end{aligned}
$$

和第一步中一样,滚动数组 dp 即可,\textbf{边界条件} $S(x,\cdot)=g(x)$。


\textbf{注意到做法二可以求出所有 $S(\left\lfloor\frac n x\right\rfloor)$,这是做法一无法完成的。}



以下是洛谷 P5325 求积性函数 $f(p^k)=p^k(p^k-1),p\in\mathbb P$ 的前缀和的代码:
\inputminted{cpp}{src/number theory/min25.cpp}
        \newpage

        \section{计算几何}
        \subsection{声明与宏}
        \inputminted{cpp}{src/geometry/define.cpp}
        \subsection{点与向量}
        \inputminted{cpp}{src/geometry/vector.cpp}
        \subsection{线}
        \inputminted{cpp}{src/geometry/line.cpp}
        \subsection{圆}
        \inputminted{cpp}{src/geometry/circle.cpp}
        \subsection{凸包}
        \inputminted{cpp}{src/geometry/convex.cpp}
        \subsection{三角形}
        \inputminted{cpp}{src/geometry/triangle.cpp}
        \subsection{多边形}
        \inputminted{cpp}{src/geometry/polygon.cpp}
        \subsection{半平面交}
        \inputminted{cpp}{src/geometry/half-plane.cpp}
        \newpage

        \section{杂项}
        \subsection{生成树计数}
        【根据度数求方案】对于给定每个点度数为 $d_i$ 的无根树,方案数为:
$$
\dfrac{(n-2)!}{\prod_{i=1}^{n}(d_i-1)!}
$$
【根据连通块数量与大小求方案】一个 $n$ 个点 $m$ 条边的带标号无向图有 $k$ 个连通块,每个连通块大小为 $s_i$,需要增加 $k-1$ 条边使得整个图联通,方案数为:(但是当 $k=1$ 时需要特判)
$$
n^{k-2}\cdot\prod_{i=1}^{k}s_i
$$
证明只需考虑 prufer 序列即可。
        \subsection{类欧几里得}
        $ax+by=n$ 的几何意义可以想象为一条直线,那么 $[0,n]$ 中可以被表示出来的整数就是 $(0,0),\left(\frac{n}{a},0\right),\left(0,\frac{n}{b}\right)$ 为顶点的三角形在第一象限内含有的整点个数。

显然的结论就是,在 $[0,n]$ 可以表示出的整数数量为:
$$
\sum_{x=0}^{\lfloor\frac{n}{a}\rfloor}\left\lfloor\dfrac{n-ax}{b}\right\rfloor
$$
类欧几里得可以在 $\mathcal{O(\log{\max(a,b)})}$ 的时间内解决此类问题。

求 $\sum_{i=0}^{n}\lfloor \frac{ai+b}{c} \rfloor$ :
\inputminted{cpp}{src/number theory/euclid1.cpp}
求 $\sum_{i=0}^{n}{\lfloor \frac{ai+b}{c} \rfloor}^2$ 和 $\sum\limits_{i=0}^{n}i\lfloor \frac{ai+b}{c} \rfloor$ ,分别对应以下的 \verb|g| 和 \verb|h| :
\inputminted{cpp}{src/number theory/euclid2.cpp}
        \subsection{没有精度问题的整除}
        \inputminted{cpp}{src/else/div.cpp}
        \newpage

        \section{其它工具}
        \subsection{编译命令}
        \inputminted{bash}{src/tools/compile.sh}
        \subsection{快读}
        \inputminted{cpp}{src/tools/fastio.cpp}
        \subsection{Python Hints}
        \paragraph*{itertools 库}

\begin{minted}[]{python3}
from itertools import *
# 笛卡尔积
product('ABCD', 'xy') # Ax Ay Bx By Cx Cy Dx Dy
product(range(2), repeat=3) # 000 001 010 011 100 101 110 111
# 排列
permutations('ABCD', 2) # AB AC AD BA BC BD CA CB CD DA DB DC
# 组合
combinations('ABCD', 2) # AB AC AD BC BD CD
# 有重复的组合
combinations_with_replacement('ABC', 2) # AA AB AC BB BC CC
\end{minted}

\paragraph*{random 库}
\begin{minted}[]{python3}
from random import * 
randint(l, r) # 在 [l, r] 内的随机整数
choice([1, 2, 3, 5, 8]) # 随机选择序列中一个元素
sample([1, 2, 3, 4, 5], k=2) # 随机抽样两个元素
shuffle(x) # 原地打乱序列 x
l,r = sorted(choices(range(1, N+1), k=2)) # 生成随机区间 [l,r]
binomialvariate(n, p) # 返回服从 B(n,p) 的一个变量
normalvariate(mu, sigma) # 返回服从 N(mu,sigma) 的一个变量
\end{minted}

\paragraph*{列表操作}
\begin{minted}[]{python3}
# 列表操作
l = sample(range(100000), 10)
l.sort() # 原地排序
l.sort(key=lambda x:x%10) # 按末尾排序
from functools import cmp_to_key
l.sort(key=cmp_to_key(lambda x,y:y-x)) # 比较函数,小于返回负数
sorted(l) # 非原地排序
l.reverse() reversed(l)
\end{minted}
\paragraph*{字典操作}
\begin{minted}[]{python3}
from collections import defaultdict
# 提供一个函数返回缺省值
d = defaultdict(list)
d["a"].append(2)
d["a"].append(3)
d["b"].append(4)
print(d) # {'a': [2, 3], 'b': [4]}
# 用 lambda 可以快速构造出需要的默认值
d = defaultdict(lambda: 2)
# 遍历键值对
for k,v in d.items():
    print(k, v)
\end{minted}

\paragraph*{复数}
\begin{minted}[]{python3}
a = 1+2j
print(a.real, a.imag, abs(a), a.conjugate())
\end{minted}

\paragraph*{高精度小数}
\begin{minted}[]{python3}
from decimal import Decimal, getcontext, FloatOperation, ROUND_HALF_EVEN
getcontext().prec = 100 # 设置有效位数
getcontext().rounding = getattr(ROUND_HALF_EVEN) # 四舍六入五成双
getcontext().traps[FloatOperation] = True # 禁止 float 混合运算
a = Decimal("114514.1919810")
print(a, f"{a:.2f}")
a.ln() a.log10() a.sqrt() a**2
\end{minted}


\paragraph*{记忆化搜索}
\begin{minted}[]{python3}
from functools import cache
# 记忆化搜索,还可以记忆化元组,只要参数满足 Hashable 即可
@cache 
def fib(n):
    if n<=2:
        return 1
    return fib(n-1)+fib(n-2)
\end{minted}
        \subsection{对拍器}
        \inputminted{python3}{src/tools/llx_checker.py}
        \subsection{常数表}
        \begin{table}[H]
\centering
\resizebox{\columnwidth}{!}{
\begin{tabular}{|l|l|l|l|l|}
\hline
\rowcolor[HTML]{C0C0C0} 
$n$ & $\log_{10} n$ & $n!$ & $C(n,n/2)$ & $\operatorname{lcm}(1...n)$ \\
2          & 0.301030   & 2              & 2              & 2              \\\hline
3          & 0.477121   & 6              & 3              & 6              \\\hline
4          & 0.602060   & 24             & 6              & 12             \\\hline
5          & 0.698970   & 120            & 10             & 60             \\\hline
6          & 0.778151   & 720            & 20             & 60             \\\hline
7          & 0.845098   & 5040           & 35             & 420            \\\hline
8          & 0.903090   & 40320          & 70             & 840            \\\hline
9          & 0.954243   & 362880         & 126            & 2520           \\\hline
10         & 1.000000   & 3628800        & 252            & 2520           \\\hline
11         & 1.041393   & 39916800       & 462            & 27720          \\\hline
12         & 1.079181   & 479001600      & 924            & 27720          \\\hline
15         & 1.176091   & 1.31e12        & 6435           & 360360         \\\hline
20         & 1.301030   & 2.43e18        & 184756         & 232792560      \\\hline
25         & 1.397940   & 1.55e25        & 5200300        & 2.68e10        \\\hline
30         & 1.477121   & 2.65e32        & 155117520      & 2.33e12        \\\hline
\end{tabular}
}
\end{table}

\begin{table}[H]
\centering
\resizebox{\columnwidth}{!}{
\begin{tabular}{|l|llllll|}
\hline
\rowcolor[HTML]{C0C0C0} 
$n\le$              & \multicolumn{1}{l|}{\cellcolor[HTML]{C0C0C0}$10^1$}  & \multicolumn{1}{l|}{\cellcolor[HTML]{C0C0C0}$10^2$}  & \multicolumn{1}{l|}{\cellcolor[HTML]{C0C0C0}$10^3$}  & \multicolumn{1}{l|}{\cellcolor[HTML]{C0C0C0}$10^4$}    & \multicolumn{1}{l|}{\cellcolor[HTML]{C0C0C0}$10^5$}  & \cellcolor[HTML]{C0C0C0}$10^6$ \\ \hline
$\max\{\omega(n)\}$ & \multicolumn{1}{l|}{2}                               & \multicolumn{1}{l|}{3}                               & \multicolumn{1}{l|}{4}                               & \multicolumn{1}{l|}{5}                                 & \multicolumn{1}{l|}{26}                              & 7                              \\ \hline
$\max\{d(n)\}$      & \multicolumn{1}{l|}{4}                               & \multicolumn{1}{l|}{12}                              & \multicolumn{1}{l|}{32}                              & \multicolumn{1}{l|}{64}                                & \multicolumn{1}{l|}{128}                             & 240                            \\ \hline
$\pi(n)$            & \multicolumn{1}{l|}{4}                               & \multicolumn{1}{l|}{25}                              & \multicolumn{1}{l|}{168}                             & \multicolumn{1}{l|}{1229}                              & \multicolumn{1}{l|}{9592}                            & 78498                          \\ \hline
\rowcolor[HTML]{C0C0C0} 
$n\le$              & \multicolumn{1}{l|}{\cellcolor[HTML]{C0C0C0}$10^7$}  & \multicolumn{1}{l|}{\cellcolor[HTML]{C0C0C0}$10^8$}  & \multicolumn{1}{l|}{\cellcolor[HTML]{C0C0C0}$10^9$}  & \multicolumn{1}{l|}{\cellcolor[HTML]{C0C0C0}$10^{10}$} & \multicolumn{1}{l|}{\cellcolor[HTML]{C0C0C0}$10^{11}$} & $10^{12}$                        \\ \hline
$\max\{\omega(n)\}$ & \multicolumn{1}{l|}{8}                               & \multicolumn{1}{l|}{8}                               & \multicolumn{1}{l|}{9}                               & \multicolumn{1}{l|}{10}                                & \multicolumn{1}{l|}{10}                              & 11                             \\ \hline
$\max\{d(n)\}$      & \multicolumn{1}{l|}{448}                             & \multicolumn{1}{l|}{768}                             & \multicolumn{1}{l|}{1344}                            & \multicolumn{1}{l|}{2304}                              & \multicolumn{1}{l|}{4032}                            & 6720                           \\ \hline
$\pi(n)$            & \multicolumn{1}{l|}{664579}                          & \multicolumn{1}{l|}{5761455}                         & \multicolumn{1}{l|}{5.08e7}                          & \multicolumn{1}{l|}{4.55e8}                            & \multicolumn{1}{l|}{4.12e9}                          & 3.7e10                         \\ \hline
\rowcolor[HTML]{C0C0C0} 
$n\le$              & \multicolumn{1}{l|}{\cellcolor[HTML]{C0C0C0}$10^{13}$} & \multicolumn{1}{l|}{\cellcolor[HTML]{C0C0C0}$10^{14}$} & \multicolumn{1}{l|}{\cellcolor[HTML]{C0C0C0}$10^{15}$} & \multicolumn{1}{l|}{\cellcolor[HTML]{C0C0C0}$10^{16}$}   & \multicolumn{1}{l|}{\cellcolor[HTML]{C0C0C0}$10^{17}$} & $10^{18}$                        \\ \hline
$\max\{\omega(n)\}$ & \multicolumn{1}{l|}{12}                              & \multicolumn{1}{l|}{12}                              & \multicolumn{1}{l|}{13}                              & \multicolumn{1}{l|}{13}                                & \multicolumn{1}{l|}{14}                              & 15                             \\ \hline
$\max\{d(n)\}$      & \multicolumn{1}{l|}{10752}                           & \multicolumn{1}{l|}{17280}                           & \multicolumn{1}{l|}{26880}                           & \multicolumn{1}{l|}{41472}                             & \multicolumn{1}{l|}{64512}                           & 103680                         \\ \hline
$\pi(n)$            & \multicolumn{6}{l|}{$\pi(x)\sim x/\ln(x)$}                                                                                                                                                                                                                                                                          \\ \hline
\end{tabular}
}
\end{table}
        \subsection{试机赛}
        \begin{itemize}
    \item 检查身份证件:护照、学生证、胸牌以及现场所需通行证。
    \item 确认什么东西能带进场。特别注意:手机、智能手表、金属(钥匙)等等。
    \item 测试鼠标、键盘、显示器和座椅。如果有问题,立刻联系工作人员。
    \item 测试编译器版本。
    \begin{itemize}
        \item  \texttt{\#include<bits/stdc++.h>}
        \item  C++20: \texttt{cin>>(s+1);}
        \item  C++17: \texttt{auto [x,y]=pair{1,"abc"};}
        \item  C++11: \texttt{auto x=1;}
    \end{itemize}
    \item 测试 \_\_int128, \_\_float128, long double
    \item 测试 pragma 是否 CE 。
    \item 测试 ‐fsanitize=address,undefined
    \item 测试本地性能与提交性能
        \begin{minted}{cpp}
        const int N=1<<20;
        for(int T=4;T;T--){
            int a[N],b[N],c[N];
            for(int i=0;i<N;i++)a[i]=i,b[i]=N-i;
            ntt_init(N);ntt(a,N,1);ntt(b,N,1);
            for(int i=0;i<N;i++)c[i]=a[i]*b[i]; // 注意不要取模谢谢
            ntt(c,N,-1);
            assert(c[0]==0xad223d);
        }
        // 本地: 291 ms
        // QOJ: 340+-40 ms
        // CF: 1000+-100 ms
        \end{minted}
    \item 测试 \verb|make| 命令。
\end{itemize}




        % \subsubsection{linux}
        % \inputminted{cpp}{src/tools/linux_checker.cpp}
        % \subsubsection{windows}
        % \inputminted{cpp}{src/tools/win_checker.cpp}
        % 构建表达式树
    \end{multicols}
\end{document}